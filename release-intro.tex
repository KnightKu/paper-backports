\subsection{The Linux kernel development model}

The Linux kernel versioning scheme consists of a version, patchlevel,
sublevel, and extraversion, all named after each attributes defined
on the project top level Makefile. A \textbf{\textit{major kernel}} release
happens when the version changes or when the version remains the same but the
patchlevel is updated. A \textbf{\textit{major kernel release}} always has an
implicit sublevel of 0. A \textbf{\textit{stable release}} follows on top of a
\textbf{\textit{major kernel release}}, both the version and patch level remain
the same but the sublevel updates. A \textbf{\textit{release candidate}}
consists of a version, patchlevel, an implicit sublevel of 0, and an
extraversion prefixed with "rc" and a digit representing the release
candidate number, starting from 1. For example the v3.16-rc1 release has a
version 3, sublevel 16, an implicit sublevel 0, and rc1 extraversion.

Development for Linux happens through a set of distributed development
trees, each development tree exclusively handles a set of components known
as \textbf{\textit{subsystems}}, the list of components each tree is
responsible for is documented on the project MAINTAINERS\footnote{https://git.kernel.org/cgit/linux/kernel/git/torvalds/linux.git/tree/MAINTAINERS} file.
Each development tree carries a delta for its components on top of the
latest known \textbf{\textit{major kernel release}} release. Each
\textbf{\textit{subsystem}} has a respective
\textbf{\textit{subsystems maintainer}} or group of
\textbf{\textit{subsystems maintainers}}. All new kernel features and
drivers are queued on \textbf{\textit{subsystems development trees}} and in order
to help make a release each \textbf{\textit{subsystem maintainer(s)}}
eventually sends a pull request of their \textbf{\textit{subsystem development tree}}
to Linus Torvalds during the \textbf{\textit{merge window}}.
The \textbf{\textit{merge window}} begins
as soon as Linus Torvalds release the next \textbf{\textit{major kernel release}}
of Linux, the \textbf{\textit{merge window}} culminates when Linus Torvalds
release the first \textbf{\textit{release candidate}} of a
new \textbf{\textit{major kernel release}}; this entire process typically
lasts about 1-2 weeks. For instance, the v3.16
\textbf{\textit{merge window}} happened in between the release of
v3.15 (June 8, 2014) and v3.16-rc1 (June 15, 2014).  A period known as the
\textbf{\textit{release candidate evaluation}} period
or \textbf{\textit{rc cycle}} follows the release of the first
\textbf{\textit{release candidate}} after the \textbf{\textit{merge window}}.
A \textbf{\textit{major kernel release}} may go through 5-9
\textbf{\textit{release candidates}}.

The purpose of the \textbf{\textit{release candidate evaluation}} period ideally
is to only find and fix regressions for new code merged since the last
\textbf{\textit{major kernel release}}. For example the
v3.16 \textbf{\textit{major kernel release}} had 7 \textbf{\textit{release candidates}},
v3.16-rc1 - v3.16-rc7, prior to Linus having made the final v3.16 \textbf{\textit{major kernel release}}.
\textbf{\textit{Subsystem maintainer(s)} queue regression fixes for the
current \textbf{\textit{release candidate}} on a \textbf{\textit{subsystem stable tree}},
used only to send pull requests to Linus during the
\textbf{\textit{release candidates evaluation}} period.
A new driver by definition could not have been present on prior major release, in this case v3.15, and
as such an exception is made to allow new device drivers during the
\textbf{\textit{release candidate evaluation}} period. A
\textbf{\textit{major kernel release}} is followed by
\textbf{\textit{stable releases}} for it, for instance v3.16 was followed
by v3.16.1, v3.16.2 and will continue to have releases until the
kernel is declared as unmaintained. \textbf{\textit{Stable releases}} only
contain critical bug fixes, they may never contain new features or drivers.

The linux-next\footnote{git://git.kernel.org/pub/scm/linux/kernel/git/next/linux-next.git}
tree is used to mimic the \textbf{\textit{merge window}}
on daily basis by pulling each \textbf{\textit{subsystems tree}} through
a set of scripts. Every day the tree is reset to the latest
\textbf{\textit{major kernel release}} and then each
\textbf{\textit{subsystems tree}} pulled, after all trees are pulled
the tree as tagged to reflect reflect the calendar date. The linux-next
tree could be used to track the latest development efforts on all
subsystems on a daily basis.
