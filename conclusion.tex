\section{Conclusions and Future Work}
\label{concl}

The Linux backports project currently supports 5 semantic patches totaling
198 lines of code.  These semantic patches affect 215 files in the
linux-next of October 15, 2014.  Recently, with v3.15, the Linux backports
project has removed support for kernels older than v3.0 (see Figure
\ref{num_patches}).  Six other semantic patches that had been developed for
supporting older kernels were removed at this time.  Furthermore, some glue
code is still implemented by direct modification to the driver code.  Using
a semantic patch is appropriate when the changes are complex, are relevant
to many drivers, and are susceptible to be affected by other evolutions in
the code.  Overall, the use of semantic patches has contributed to the
robustness of the backport process.  Indeed, another developer on the
backports project recently stated ``All the patches that broke often in the
early days are now using coccinelle or are removed because they were only
needed for the older kernel versions.''\footnote{Hauke Mehrtens, private
  email of October 23, 2014.}

The work on backports raises a number of directions for future work.  One
direction would be to reduce the need for glue code by integrating upstream
static inline functions for accessing and updating key data structures.
Patches to address this have been submitted and have now started to be
accepted at least on the networking subsystem, maintained by David Miller,
specifically to help reduce the amount of work to backport the ieee802154
subsystem.\footnote{https://lkml.org/lkml/2014/4/17/663} this change is now
merged as part of the v3.18-rc1 release.

Another direction would be to infer semantic patches.  For many of our
backports, we have a collection of manually written patches that make the
same change.  Backporting could be further streamlined by inferring
semantic patches from these examples.  Preliminary work has indeed been
done on the automatic inference of change specifications
\cite{jAndersenAse2008,Meng:13}.  Alternatively, we observe that our
additions of glue code amount to (the inverses of) collateral evolutions.
If a library change is accompanied by a semantic patch, to ease updating
the library's clients, then it might be possible to systematically invert
this semantic patch for subsequent backporting.  Another direction would be
to infer the glue code itself.

Finally, it is always a concern that a change in the code may break
semantic invariants.  We leave as future work to investigate whether change
impact analysis, as described above, can be relevant here.

\subsection{License}
This paper is licensed under the Creative Commons BY-SA 4.0.
%% \subsection{Acknowledgements}

%% A lot of the work put forward on backports with Coccinelle integration
%% was made possible thanks to funding by INRIA and IRILL.
