Software is continually evolving, to fix bugs and add new features.
Industry users, however, often value stability, and thus are not always
able to update their code base to the latest versions.  This raises the
need to selectively backport new features to older software versions.
Traditionally, backporting has been done by cluttering the backported code
with preprocessor directives, to replace behaviors that are unsupported in
an earlier version by appropriate workarounds.  This approach however
involves writing a lot of error-prone backporting code, and results in
implementations that are hard to read and maintain.  We consider this issue
in the context of the Linux kernel, for which older versions are in wide
use.  We present a new backporting strategy that relies on the use of a
compatability library and on code that is automatically generated using the
program transformation tool Coccinelle.  This approach reduces the amount
of code that must be manually written, and thus can help the Linux kernel
backporting effort scale.
