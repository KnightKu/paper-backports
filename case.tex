\section{Case Study}
\label{case}

To test the limits of what can be backported using Coccinelle, we chose the
most complex collateral evolution supported by the backports project as a
test case.  Specifically, we decided to try to backport threaded IRQ
support, introduced in the v2.6.31 kernel.  This backport requires
modifications to a driver-specific structure, as well as to multiple driver
functions.

\subsection{Backporting threaded IRQ support the old way}

We first explain how the backports project provided backport support for
threaded IRQ support prior to the use of Coccinelle.  The first step was to
extend the compat library with support for threaded IRQs, as shown in
Figure \ref{compat}.  This involved creating a new data type {\tt
  compat\_\-threa\-ded\_\-irq} (lines 1-11) to collect some extra information
for each driver, and creating a set of helper functions to implement the
threaded IRQ fuctionality (lines 13-69).  The helper functions include {\tt
  compat\_\-request\_\-threa\-ded\_\-irq} (lines 26-44), which initializes
the fields of the {\tt compat\_\-threa\-ded\_\-irq} structure and then calls
{\tt request\_\-irq}, and functions such as {\tt
  compat\_\-free\_\-threa\-ded\_\-irq} (lines 46-51) and {\tt
  compat\_\-synchronize\_\-threa\-ded\_\-irq} (lines 62-69) that call their
unthreaded counterparts on information stored in the {\tt
  compat\_\-threa\-ded\_\-irq} structure.  In all, the extension to the
compat library amounts to 75 lines of code.\footnote{Computed using David
  Wheeler's SLOCCount, \newline http://www.dwheeler.com/sloccount/.}

%% 10 for structure, 65 for functions
%% The complete code is in compat_extend.c

\begin{figure}[h!]
\begin{lstlisting}[language=C]
#if LINUX_VERSION_CODE < KERNEL_VERSION(2,6,31)
struct compat_threaded_irq {
	unsigned int irq;
	irq_handler_t handler;
	irq_handler_t thread_fn;
	void *dev_id;
	char wq_name[64];
	struct workqueue_struct *wq;
	struct work_struct work;
};
#endif

#if LINUX_VERSION_CODE < KERNEL_VERSION(2,6,31)
static inline
void compat_irq_work(struct work_struct *work)
{
  ...
}

static inline
irqreturn_t compat_irq_dispatcher(int irq, void *dev_id)
{
  ...
}

static inline
int compat_request_threaded_irq(
	struct compat_threaded_irq *comp,
	unsigned int irq,
	irq_handler_t handler,
	irq_handler_t thread_fn,
	unsigned long flags,
	const char *name,
	void *dev_id)
{
  comp->irq = irq;
  comp->handler = handler;
  comp->thread_fn = thread_fn;
  comp->dev_id = dev_id;
  INIT_WORK(&comp->work, compat_irq_work);
  ...
  return request_irq(irq, compat_irq_dispatcher, flags,
		 name, comp);
}

static inline
void compat_free_threaded_irq(
	struct compat_threaded_irq *comp)
{
  free_irq(comp->irq, comp);
}

static inline
void compat_destroy_threaded_irq(
	struct compat_threaded_irq *comp)
{
  if (comp->wq)
    destroy_workqueue(comp->wq);
  comp->wq = NULL;
}

static inline
void compat_synchronize_threaded_irq(
	struct compat_threaded_irq *comp)
{
  synchronize_irq(comp->irq);
  cancel_work_sync(&comp->work);
}
#endif
\end{lstlisting}
\caption{Extensions to the compat library to support \newline {\tt
    request\_\-threa\-ded\_\-irq}}
\label{compat}
\end{figure}

Each driver to backport that relies on threaded IRQs then needs to be
modified to make use of the new helper functions.  Figure \ref{compatpatch}
shows the modifications for the b43 driver.  Lines 1-13 update a header
file to extend the driver's private {\tt b43\_\-wldev} structure type with
a field containing the compat structure when the kernel version is lower
than the first one that supports threaded irqs.  Lines 14-52 replace each
threaded IRQ operation with its compat version, again for kernels for which
threaded irqs are not already supported.

\begin{figure}
\begin{lstlisting}[language=diff]
--- a/drivers/net/wireless/b43/b43.h
+++ b/drivers/net/wireless/b43/b43.h
@@ -805,6 +805,9 @@ enum {
 
 /* Data structure for one wireless device (802.11 core) */
 struct b43_wldev {
+#if LINUX_VERSION_CODE < KERNEL_VERSION(2,6,31)
+	struct compat_threaded_irq irq_compat;
+#endif
 	struct b43_bus_dev *dev;
 	struct b43_wl *wl;
 	/* a completion event structure needed if this call is
            asynchronous */
--- a/drivers/net/wireless/b43/main.c
+++ b/drivers/net/wireless/b43/main.c
@@ -4243,8 +4243,17 @@ redo:
 	if (b43_bus_host_is_sdio(dev->dev)) {
 		b43_sdio_free_irq(dev);
 	} else {
+#if LINUX_VERSION_CODE >= KERNEL_VERSION(2,6,31)
 		synchronize_irq(dev->dev->irq);
+#else
+		compat_synchronize_threaded_irq(&dev->irq_compat);
+#endif
+#if LINUX_VERSION_CODE >= KERNEL_VERSION(2,6,31)
 		free_irq(dev->dev->irq, dev);
+#else
+		compat_free_threaded_irq(&dev->irq_compat);
+		compat_destroy_threaded_irq(&dev->irq_compat);
+#endif
 	}
 	mutex_lock(&wl->mutex);
 	dev = wl->current_dev;
@@ -4290,9 +4299,17 @@ static int b43_wireless_core_start(
 			goto out;
 		}
 	} else {
+#if LINUX_VERSION_CODE >= KERNEL_VERSION(2,6,31)
                err = request_threaded_irq(dev->dev->irq,
				b43_interrupt_handler,
				b43_interrupt_thread_handler,
				IRQF_SHARED, KBUILD_MODNAME, dev);
+#else
+               err = compat_request_threaded_irq(&dev->irq_compat,
+					dev->dev->irq,
+					b43_interrupt_handler,
+					b43_interrupt_thread_handler,
+					IRQF_SHARED, KBUILD_MODNAME, dev);
+#endif
 		if (err) {
 			b43err(dev->wl, "Cannot request IRQ-%d\n",
 			       dev->dev->irq);
\end{lstlisting}
\caption{Backporting the b43 driver}
\label{compatpatch}
\end{figure}

The changes shown in Figure \ref{compatpatch} amount to 20 lines of added
code and only apply to a single driver.  As of the linux-next of October
15, 2014, 169 files contain at least one call to {\tt
  request\_\-threa\-ded\_\-irq}, and of these 16 are in subsystems supported
by the Linux backports project.\footnote{1 in ethernet, 7 in wireless, 0 in
  bluetooth, 2 in nfc, 0 in ieee802145, 4 in media, 2 in regulator.}
Backporting all of the 169 files that use threaded IRQs to Linux versions
prior to v2.6.31 would require developing and maintaining over 3000 lines
of patch code.

\subsection{Backporting threaded IRQ support with Coccinelle}

Figure \ref{thrdirq} shows a Coccinelle semantic patch that automates these
changes.  Most are relatively trivial: replace one call with another,
with the new call using the backport data structure among its arguments.  A
typical example is illustrated in the first rule (lines 1-24), where a call
to {\tt request\_\-threa\-ded\_\-irq} is replaced by a call to {\tt
  compat\_\-request\_\-threa\-ded\_\-irq}.  The new call takes the same
arguments as the old one, with the addition of the
first argument (line 17), which is the {\tt compat\_\-threa\-ded\_\-irq}
structure.

\begin{figure}
\begin{lstlisting}[language=diff]
@ threaded_irq @
identifier ret;
expression irq, irq_handler, irq_thread_handler, flags,
           name;
type T;
T *private;
@@

+#if LINUX_VERSION_CODE >= KERNEL_VERSION(2,6,31)
ret = request_threaded_irq(irq,
			   irq_handler,
			   irq_thread_handler,
			   flags,
			   name,
			   private);
+#else
+ret = compat_request_threaded_irq(&private->irq_compat,
+				   irq,
+				   irq_handler,
+				   irq_thread_handler,
+				   flags,
+				   name,
+				   private);
+#endif

@ sync_irq depends on threaded_irq @
expression irq;
type threaded_irq.T;
T *threaded_irq.private;
@@

+#if LINUX_VERSION_CODE >= KERNEL_VERSION(2,6,31)
synchronize_irq(irq);
+#else
+compat_synchronize_threaded_irq(&private->irq_compat);
+#endif

@ free depends on threaded_irq @
expression irq, dev;
type threaded_irq.T;
T *threaded_irq.private;
@@

+#if LINUX_VERSION_CODE >= KERNEL_VERSION(2,6,31)
free_irq(irq, dev);
+#else
+compat_free_threaded_irq(&private->irq_compat);
+compat_destroy_threaded_irq(&dev->irq_compat);
+#endif

@ modify_private_header depends on threaded_irq @
type threaded_irq.T;
@@

T {
+#if LINUX_VERSION_CODE < KERNEL_VERSION(2,6,31)
+      struct compat_threaded_irq irq_compat;
+#endif
...
};
\end{lstlisting}
\caption{Backporting threaded IRQs with Coccinelle}
\label{thrdirq}
\end{figure}

A challenge in implementing this backport is where to store the {\tt
  compat\_\-threa\-ded\_\-irq} structure.  As multiple instances of a device
may be present in a running kernel, this structure cannot simply be a
global variable of the device driver.  In the case of the b43 driver, we
placed this structure into the driver's {\tt b43\_\-wldev} structure.  To
generalize this, we need to find a suitable location for this structure in
each driver to which the semantic patch may be applied.

The need to support multiple instances of a data structure at run time is a
common problem in device driver development, and the Linux kernel proposes
a standard solution, the use of a {\em private} data structure.  An
instance of this private structure is created when a device is initialized,
and then the driver infrastructure makes this structure available to each
driver callback function, much like the implicit ``this'' argument found in
object-oriented languages.  Normally, each driver defines a specific
private-structure type, containing the information that is specific to its
operation.  We exploit this private structure to store the {\tt
  compat\_\-threa\-ded\_\-irq} structure.

To use the private structure to store the {\tt compat\_\-threa\-ded\_\-irq}
structure, we must address two issues.  First, for each driver, we must
find the type of the private structure and extend the corresponding type
declaration with a field for the {\tt compat\_\-threa\-ded\_\-irq} structure.
Second, we must find the name of the current instance of the private
structure at each point where a reference to the {\tt
  compat\_\-threa\-ded\_\-irq} structure is needed for the backport process.

To address the first issue, we exploit the fact that Coccinelle collects
type information when analyzing the source code, and makes it possible to
manipulate this type information via metavariables in a semantic patch.
Fortunately, device drivers typically already pass their private structure
as the last argument to {\tt request\_\-threa\-ded\_\-irq}, as the
information contained in the private structure is typically also needed by
the interrupt handler, which is installed by this function.  By matching
this reference to the private structure, Coccinelle makes it possible to
obtain its type.  Concretely, line 5 of Figure \ref{thrdirq} declares a
type metavariable {\tt T}, which is then used in describing the type of
metavariable {\tt private}.  Matching {\tt private} against the code in the
last argument of {\tt request\_\-threa\-ded\_\-irq} has the side effect of
storing the type of the matched code in {\tt T}, where it can be used by
subsequent rules.  In the fourth rule (lines 51 to 60), {\tt T}, referenced
as {\tt threaded\_\-irq.T}, is used to match and extend the definition of
the private structure, adding a new field {\tt irq\_\-compat} to the
beginning of the private structure when the kernel version is less than
v2.6.31.

To address the second issue, we exploit the fact that, within a given
driver, the Linux developers typically give the private structure the same
name, in every function in which it is used.  Thus, we simply inherit the
term matched by the metavariable {\tt private} defined in the rule {\tt
  threaded\_\-irq}, and use that term in the added calls in the {\tt
  synch\_\-irq} and {\tt free} rules.  This solution is not safe, but it is
pragmatic, in that it simplifies the transformation and exploits properties
of the Linux coding style.  Note that in the {\tt free\_\-irq} case (lines
38-49), we could also use the second argument to {\tt free\_\-irq}, which
by definition of the IRQ API should point to the same structure as the last
argument to {\tt request\_\-threa\-ded\_\-irq}.\footnote{Actually,
  unintentionally, in the second call that is added by this rule, this
  strategy is used.}  This value is not immediately available in the {\tt
  sync\_\-irq} rule (lines 26-36), but we could extend the rule to match a
neighboring expression of the right type, for a safer solution.

\subsection{Benefits of using Coccinelle for backporting}

Reimplementing the threaded IRQ backport using Coccinelle revealed that the
original manual backport was inconsistent.  Specifically, in the manual
backport, the compat structure, {\tt compat\_\-thread\_\-irq}, was integrated
into different structures in different drivers.  Backporting is
intrinsically risky, because the older code may not respect the invariants
required by the backported code.  Modifying the older code in a consistent
way reduces the set of issues that can arise.  Doing so also
makes the results easier for developers to understand.

Reimplementing the threaded IRQ backport using Coccinelle also revealed
that the existing threaded IRQ patch series also backported another
collateral evolution, related to the management of IRQ flags.  Isolating
each collateral evolution in a separate patch benefits the backports
project, by making it easier to understand how to backport other drivers,
which may need only one of the changes.  Using Coccinelle not only makes
the need for this split apparent, it also makes it easier to manage the
resulting set of changes.  While two sets of changes are now needed, each
is performed by a single semantic patch that can be applied to many files,
rather than having to implement and record each of the changes
individually.

Finally, for a few patch series, the amount of time to generate a
backported release was actually reduced, as compared to the application of
the manually created patches.  Coccinelle has to parse the driver C code
and search for relevant code, all of which are much more expensive than
applying a patch, in which the file name, line number, and code context are
explicitly specified.  Nevertheless, patch application is sequential, while
Coccinelle can work on a collection of individual files in parallel.  This
difference is sufficient to make the use of Coccinelle faster in some
cases.

