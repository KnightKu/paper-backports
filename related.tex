\section{Related Work}
\label{related}

The specific problem of backporting has not received much attention in the
research community.  Backporting is, however, related to more general
issues of change management, as arise when merging trees in a source code
management system and when integrating changes developed for one branch of
a software product line into another branch.

Uquillas G\'omez et al.~\cite{UquillasGomez:wcre10} propose visualization
tools to aid the developer in integrating a patch developed for one branch
of a software project into another branch of the software project.  They
focus on individual changes and on helping the developer to identify semantic
issues that may affect the correctness of the change in the new context.
Other work on change impact analysis includes that of Gallagher and Lyle
\cite{Gall91a}, who use program slicing \cite{Weis81a} to collect
information about the impact of a change, and the tool Chianti, which
identifies change impact based on the results of test cases \cite{Ren04a}.
The work on change impact is complementary with ours.  In our case, the
correct backport is already identified, and we are concerned with expressing
it in a concise and robust way.  In the future, we could combine change
impact analysis with our approach, to further check the correctness of the
backported code.

Fiuczynski et al.\ faced the challenge of keeping an externally maintained
patchset up to date with the evolutions in the Linux kernel
\cite{Fiuczynski:hotos05}.  They proposed a preliminary solution based on
aspect-oriented programming \cite{Kiczales:01} to re-express these patches
in a more generic and robust way.  To the best of our knowledge, this tool
remained in a prototype stage.
